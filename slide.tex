\documentclass[
% draft,          % 草稿模式
aspectratio=169,  % 使用 16:9 比例
]{ctexbeamer}
\mode<presentation>

\usepackage{graphicx} %插入图片的宏包
\usepackage{float} %设置图片浮动位置的宏包
\usepackage{subfigure} %插入多图时用子图显示的宏包
\usetheme[blue,max]{sjtubeamer}
% 使用 maxplus/max/min 切换标题页样式
% 使用 red/blue 切换主色调
% 使用 light/dark 切换亮/暗色模式
% 使用外样式关键词以获得不同的边栏样式
%   miniframes infolines  sidebar 
%   default    smoothbars split	 
%   shadow     tree       smoothtree
% 使用 topright/bottomright 切换徽标位置
% 使用逗号分隔列表以同时使用多种选项

% \tikzexternalize[prefix=build/]
% 如果您需要缓存 tikz 图像,请取消注释上一行,并在编译选项中添加 -shell-escape。

\usepackage[backend=biber,style=gb7714-2015]{biblatex}
\addbibresource{thesis.bib}

\institute[SJTU]{上海交通大学 2025级ACM班} % 组织

\title{手写数字识别}         % 标题
\subtitle{卷积神经网络的一些应用}       % 副标题
\author{WyEhNf}                       % 作者
\date{\today}                        % 日期  

\begin{document}
	
	\maketitle                           % 创建标题页
	
	\part{CNN的工作模式}
	
	% 使用节目录
	\AtBeginSection[]{
		\begin{frame}
			% \tableofcontents[currentsection,hideallsubsections]  % 传统节目录             
			\sectionpage                        % 节页
		\end{frame}
	}
	
	\section{基本结构}
	
	\begin{frame}
		\frametitle{结构}
		CNN网络的主要结构如下:
		\begin{itemize}
			\item 输入层
			\item 卷积层
			\item 池化层
			\item 输出层 
		\end{itemize}
	\end{frame}
	
	\begin{frame}
		在单次训练中,CNN网络主要完成以下流程:
		\frametitle{运行流程}
		\begin{itemize}
		\item 接受输入向量
		\item 前向传播
		\item 计算梯度
		\item 反向传播
		\end{itemize}
	\end{frame}

	
	\section{各层实现}
	
	\begin{frame}
		\frametitle{卷积神经元}
		考虑一般的神经元,对于输入向量$X$,其转换函数为:
		$$
		f(x)=\sum{x_i·w_i}+b
		$$
		而卷积神经元相当于将向量间的内积变为权值矩阵与输入向量/矩阵的卷积,即对于输入向量$X$,其转换函数为:
		$$
		f(x)=W^T*X+b
		$$
		这对二维图像明显有这个更好的特征反映能力。
	\end{frame}
	
	\begin{frame}
		\frametitle{激活函数}
		常见的激活函数有两种:
		\begin{block}{Relu函数}
		$$
		f(x)=x[x>0]
		$$
		\end{block}
		\begin{block}{Sigmoid函数}
			$$
			f(x)=\dfrac{1}{1+\mathbf{exp}(-x)}
			$$
		\end{block}
		激活函数的目的是将卷积这一线性操作非线性化。
	\end{frame}
	
	\begin{frame}
		\frametitle{卷积核的移动与边界填充}
		注意到卷积核在移动的时候对边界信息的获取是弱的,一般由以下填充模式:
		\begin{block}{Valid模式}
			不进行额外填充,$r*c$的矩阵经卷积后输出为$(r-n+1)*(c-m+1)$的矩阵。
		\end{block}
			\begin{block}{Full模式}
			充分填充使得卷积核中心经过输入矩阵每个位置,$r*c$的矩阵经卷积后输出为$(r+n-1)*(c+m-1)$的矩阵。
		\end{block}
			\begin{block}{Same模式}
			适当填充使得输出矩阵大小不变。
		\end{block}
	\end{frame}
	
	

		\begin{frame}
		\frametitle{池化层}
		池化层的目的是降低进入后续层的数据规模,仅保留每个局部的整体特征。
		池化窗口的移动和卷积核类似,但不存在填充的问题。
		一般可以取窗口内平均或者max/min。
	\end{frame}
	
	
	
	\begin{frame}
		\frametitle{全连接层}
		全连接层相当于对所得到的数据的总结,可以看作是一个收束的神经元,将数据数量约束到最终期望的规模,
		
		特别的,全连接层的激活函数需要将数据转化为可供与标签比较的量。这个激活后的输出即为最终输出。
	\end{frame}
	
	\part{CNN在数字识别中的应用}
		% 使用节目录
	\AtBeginSection[]{
		\begin{frame}
			% \tableofcontents[currentsection,hideallsubsections]  % 传统节目录             
			\sectionpage                        % 节页
		\end{frame}
	}
	\section{网络搭建}
	\begin{frame}
		\frametitle{5层CNN网络}
		MINST数据集中摹刻数字的灰度矩阵是28*28的,最终对每张图片的判断应该是一个其为0-9的概率。
		\begin{itemize}
			\item 卷积层(5*5,6核)输出6张24*24的图片(选用Relu函数激活)
			\item 池化层(2*2,max)输出6张12*12的图片
			\item 卷积层(5*5,12核)输出12张8*8的图片
			\item 池化层(2*2,max)输出12张4*4的图片
			\item 全连接层 接受一个192维向量并转换为一个10维的描述概率的向量,并评估误差
		\end{itemize}
	\end{frame}
	\begin{frame}
		\frametitle{全连接层的激活函数}
		这里采用softmax函数,这个函数可以将一组数据归一化到$(0,1)$,且恰好能表示概率分布。
		\begin{block}{Softmax函数}
		$$
		\mathbf{Softmax(y_i)}=\dfrac{\mathbf{exp(y_i)}}{\sum{\mathbf{exp(y_j)}}}
		$$
		\end{block}
		
	\end{frame}
	
	\begin{frame}
		\frametitle{误差评估}
	
		每次评估误差后根据误差修正卷积核的参量。
			\begin{block}{交叉熵损失函数}
				$$
				E=-\sum{p_i}lnY_i
				$$
			\end{block}
		其中$p_i$是真实概率,$Y_i$是$\mathbf{Softmax}$函数所得。
	\end{frame}
	
	\section{误反向传播}
	\begin{frame}
		\frametitle{输出层-Softmax函数的梯度}
		$$
		\dfrac{\partial{Y_j}}{\partial{y_i}}=	\dfrac{\mathbf{exp}(y_i)(\sum{ \mathbf{exp}(y_j)-\mathbf{exp}(y_i)})}{(\sum{\mathbf{exp}(y_j)})^2}=Y_i(1-Y_i)\quad (i=j)
		$$
		$$
		\dfrac{\partial{Y_j}}{\partial{y_i}}=	\dfrac{\mathbf{exp}(y_j)(0-\mathbf{exp}(y_i))}{(\sum{\mathbf{exp}(y_j)})^2}=-Y_i·Y_j \quad(i\neq j)
		$$
		$$
		grad_i=\dfrac{\partial{E}}{\partial{y_i}}=\dfrac{\partial{(\sum{E_j})}}{\partial{y_i}}=\sum{\dfrac{\partial{E_j}}{\partial{Y_j}}\cdot \dfrac{\partial{Y_j}}{\partial{y_i}}}=-\dfrac{p_i}{Y_i}Y_i(1-Y_i)+\sum_{j \neq i}{-\dfrac{p_j}{Y_j}Y_jY_i}
		$$
		结合$\sum p_i =1$,可以得到:
		$$
		grad_i=Y_i-p_i
		$$
	\end{frame}
	
	\begin{frame}
		\frametitle{输出层-全连接神经元的梯度}
		方便起见,此后我们把上一层(也就是正向传播的下一层)所得梯度成为$pre_i$
	
		$$
		grad_i=\dfrac{\partial{E}}{\partial{y_i}}=\sum_{j=0} ^{9}{\dfrac{\partial{E_j}}{\partial{y_j}}\cdot \dfrac{\partial{y_j}}{\partial{x_i}}}
		=\sum_{j=0} ^{9}{pre_j \cdot w_{ij}}
		$$
		其中$w_{ij}$为全输出层神经元将$x_i$贡献到$y_j$的权重。
	\end{frame}
	
		\begin{frame}
		\frametitle{池化层}
		在全连接层中我们得到了12个4*4矩阵的梯度,那么经过池化层的反向传播后,矩阵扩展为原本的8*8的大小.
		由于这里池化采取的是取max,所以反向传播的时候只要维持好这个性质就可以了。
		\\
		一种可行的构造是维持池化窗口中最大值位置上的数不变,其余位置填上0。	
	\end{frame}
	

	\begin{frame}
	\frametitle{卷积层}	
	显然Relu函数的导数为:$Relu'(x)=[x>0]$。
	\\
	结合矩阵卷积的求导方法,有:
	$$
	grad_i=\sum {(pre_i \cdot {Relu'(y_i)})*w'_{ij}}
	$$
	其中$w'_{ij}$为将输入的第$j$张图片贡献到输出第$i$张图片的权值矩阵的旋转矩阵。
	前两层的梯度推导是类似的。
	\end{frame}
		\begin{frame}
		\frametitle{参数更新}	
		我们已经求出每一层的输出$y_{i}$关于$E$的梯度。
		鉴于每一层的转换都形如:
		$$
		y_i=w_{ij}x_j+b_i
		$$
		显然有:
		$$
		\dfrac{\partial{E}}{\partial{w_{ij}}}=\dfrac{\partial{E}}{\partial{y_i}}\cdot x_j
		$$
		$$
		\dfrac{\partial{E}}{\partial{b_{i}}}=\dfrac{\partial{E}}{\partial{y_i}}
		$$
		据此我们可以更新任意参量$X$:
		$$
			X'=X-\alpha \cdot grad_X
		$$
		其中$\alpha$为学习率。
	\end{frame}
	\section{实际训练}
	\begin{frame}
	\frametitle{训练参数与结果}	
	\begin{itemize}
	\item 学习率:初始设置为0.03,随着训练的进行逐渐递减到0.001,避免在训练后期步长过大跳过最优点
	\item 训练集:MINST训练集提供了60000份样本,出于时间考量,循环训练5次。
	\item 测试集:10000份与训练集格式相同的样本
	\end{itemize}
	训练结果:成功率约为98.2\%
	\end{frame}
	\section{不同模型的比较}
		\begin{frame}
		\frametitle{卷积核的数量}	
		\begin{itemize}
			\item 双卷积层(6核,12核):成功率98.2\%
			\item 双卷积层(12核,16核):成功率98.5\%
			\item 双卷积层(16核,32核):成功率98.7\%
		\end{itemize}
	\end{frame}
	
	\begin{frame}
		\frametitle{卷积层数}	
		\begin{itemize}
			\item 双卷积层(6核,12核):成功率98.2\%
			\item 三卷积层(6核,12核,16核(1*1)):成功率96.5\%
			\item 三卷积层(16核,32核,32核(3*3):成功率97.8\%
		\end{itemize}
	\end{frame}
	
	\begin{frame}
		\frametitle{对数据的重复训练}	
			
				\begin{figure}[H] %H为当前位置,!htb为忽略美学标准,htbp为浮动图形
					\centering %图片居中
					\subfigure[Without Dropout]{
					\includegraphics[width=0.27\textwidth]{test1.png}}
					\subfigure[With Dropout]{
						\includegraphics[width=0.2\textwidth]{test2.png}} %插入图片,[]中设置图片大小,{}中是图片文件名}
					%最终文档中希望显示的图片标题
					\label{Fig.main2} %用于文内引用的标签
				\end{figure}
		可以看到由于数据的复杂性较低,dropout层对减缓过拟合的作用甚至不如其丢失数据对准确性造成的影响大。
	\end{frame}
	
	\part{参考资料}
	\begin{frame}
		资料链接:
		\begin{itemize}
		\item \href{https://zhuanlan.zhihu.com/p/122559611}{大话卷积神经网络}
		\item \href{https://www.cnblogs.com/weijiakai}{weijiakai's blog}(C++/OpenCV实现CNN系列)
		\end{itemize}
	\end{frame}
	\makebottom       % 创建结束页
	
	
\end{document}